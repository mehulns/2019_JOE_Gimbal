\section{Introduction}
\label{sec:intro}

Exploration of Manganese crusts (Mn crusts) has gained signifant importance in recent years for their economic and scientific value. 

Mn crusts are found abundantly along the slopes of underwater seamounts. 

For thorough understating of these crusts, estimating their volume and distribution is important.

This  is usually carried out by using camera systems mounted on Remotely Operated Vehicles (ROVs) and sampling using cutting tools mounted on manipulators. 

The spatial resolution obtained from sampling does not provide enough details for volumetric estimation. 

For this, an acoustic sub-bottom probe was developed by our group for in-situ continuous measurements of Mn crusts thickness. 

The acoustic probe focuses a 200 kHz acoustic beam with a diameter of 20 mm at a target 1.5 m away. 

Using the reflected acoustic signal from the surface and the back face of the crust, the thickness of crusts is calculated from the time difference between the two reflections. 

Previous experiments performed by our group in a tank showed that the quality of acoustic reflection is significantly affected by the angle of incidence of the acoustic beam as seen in Figure 1. 

To make practical use of the probe during surveys, an actively controlled double gimbal system was made with a control algorithm which uses the seafloor slope to align the probe parallel to the seafloor. 

The system has been implemented on the survey class Autonomous Underwater Vehicle (AUV) BOSS-A which is a payload oriented vehicle with sensors built specifically for Mn crust surveys. 
It is equipped with the acoustic probe mounted on a double gimbal system as its main payload. 
A high resolution mapping system, capable of centimeter order seafloor reconstructions in color, is implemented on the AUV based on light sectioning. 

These reconstructions can be used to measure accurate slopes of the seafloor. 

However, for actively controlling the gimbals in real-time, a simplified algorithm is made which calculates the plane made by two points on the laser line and range to seafloor measured using the Doppler Velocity Log (DVL). 

The performance of this system during sea trials is elaborated in this paper


Organisation of the paper:

Section 2: Describes the problem with angular reflection 

Section 3: Describes the hardware associated. The acoustic probe, gimbal mechanism, visual mapping system and the control mechanism.

Section 4: Describes the method of analysis of the performance of the system

Section 5: Describes the survey performed in the ocean using this system and analyses the performance of the system.

Section 6: Provides conclusions to the performance of the system.